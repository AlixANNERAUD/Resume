\documentclass[10pt,a4paper,ragged2e,withhyper]{../AltaCV/altacv}

\geometry{left=1.25cm,right=1.25cm,top=1.5cm,bottom=1.5cm,columnsep=1.2cm}
 
\usepackage{paracol}

\ifxetex
  \setmainfont{Roboto Slab}
  \setsansfont{Lato}
  \renewcommand{\familydefault}{\sfdefault}
\else
  \usepackage[rm]{roboto}
  \usepackage[defaultsans]{lato}
  \renewcommand{\familydefault}{\sfdefault}
\fi

\definecolor{LightGrey}{HTML}{000000} 
\definecolor{Blue}{HTML}{1F7F9F}
\definecolor{Purple}{HTML}{3C3744}
\definecolor{Amber}{HTML}{FFBF00}

\colorlet{name}{Blue}
\colorlet{tagline}{Blue}
\colorlet{heading}{Blue}
\colorlet{headingrule}{Amber}
\colorlet{subheading}{Blue}
\colorlet{accent}{Blue}
\colorlet{emphasis}{Blue}
\colorlet{body}{Purple}

\renewcommand{\namefont}{\Huge\rmfamily\bfseries}
\renewcommand{\personalinfofont}{\footnotesize}
\renewcommand{\cvsectionfont}{\LARGE\rmfamily\bfseries}
\renewcommand{\cvsubsectionfont}{\large\bfseries}

\renewcommand{\cvItemMarker}{{\small\textbullet}}
\renewcommand{\cvRatingMarker}{\faCircle}

\begin{document}
\name{Alix ANNERAUD}
\tagline{Étudiant en informatique de 21 ans à l'INSA de Rouen Normandie.}

\photoL{2.5cm}{../Images/Alix.png}

\personalinfo{
  \printinfo{\faGlobe}{\url{https://alix.anneraud.fr}}
  \email{alix@anneraud.fr}
  \phone{+33 7 81 85 08 69}
  \github{AlixANNERAUD}
  \linkedin{Alix-ANNERAUD}
  %\location{Rouen, France}\\
}

\makecvheader

\columnratio{0.6}

\begin{paracol}{2}
  \cvsection{\faUser À propos}

  \begin{quote}
    ``Actuellement à la recherche d'un \textbf{stage en informatique à l'étranger} d'au moins 10 semaines entre mi-mai et août 2025 pour acquérir une expérience internationale et améliorer mes compétences.
    Je cherche également un \textbf{contrat de professionnalisation} à partir de septembre 2025.``
  \end{quote}

  \cvsection{\faBriefcase Expérience}

  \cvevent{Stagiare en recherche}{Université de New York (NYU)}{Juin 2024 -- Août 2024}{New York, USA}
  Projet de localisation de sources de gaz (GSL) avec capteurs mobiles et fixes.

\begin{itemize}
  \item \textbf{Simulations de mécanique des fluides} et \textbf{algorithmes de localisation}.
  \item \textbf{Traitement du signal} et \textbf{analyse de données} en Python.
  \item Développement de dispositifs IoT en C++.
\end{itemize}
  \divider

  \cvevent{Assistant au service des relations entreprises}{INSA de Rouen Normandie}{Septembre 2021 -- Juin 2024}{Rouen, France}

  \textbf{Rôle polyvalent d'assistant} axé sur la gestion de données, la prise de contact avec des entreprises, la gestion des dossiers étudiants et les enquêtes auprès des anciens pour la certification des cours.

  \divider

  \cvevent{Stage sur ligne de production}{Renault - Usine de Cléon}{Juin 2022 (1 mois)}{Cléon, France}

  Assemblage de stators et rotors pour moteurs électriques.

  \smallskip

  \cvsection{\faLightbulb Projets}

  \cvevent{Xila}{\url{https://xila.dev}}{}{}
    Développement d'un système d'exploitation simple pour microcontrôleurs, initialement en C++.
    Réécriture partielle en Rust pour améliorer les performances et et la sécurité.

  \divider

  \cvevent{Cartographie évidentielle (EOGM)}
  {\url{https://github.com/AlixANNERAUD/Evidential_occupancy_map}}
  {}
  {}

  À partir des recherches de notre professeur sur l’application de la théorie de Dempster-Shafer en robotique, nous avons développé un algorithme en temps réel qui génère une carte d'occupation via des capteurs LiDAR, IMU et GPS embarqués.
  Conçu en C++ pour ROS, il est optimisé pour la rapidité et l’efficacité grâce à des techniques d'optimisation de mémoire, vectorisation et parallélisation des calculs.

  \medskip

  \switchcolumn

  \cvsection{\faGraduationCap Formation}

  \cvevent{Diplôme d'ingénieur en informatique}{INSA de Rouen Normandie}{Sept 2021 -- En cours}{Rouen, France}

  \divider

  \cvevent{Baccalauréat scientifique (section européenne)}{Lycée Jean-Paul II}{2021}{Rouen, France}
  Mention très bien.

  \divider

  \cvevent{Cambridge B1 Preliminary}{}{2018}{}

  \smallskip

  \cvsection{\faThumbsUp Loisirs}

  \cvachievement{\faMicrochip}{Informatique et électronique}{Passionné par l'informatique et l'électronique depuis l'enfance, je souhaite devenir ingénieur dans ce domaine.}

  \divider

  \cvachievement{\faSubway}{Modélisme}{Construction d'un réseau de trains miniatures avec mon père.}

  \divider

  \cvachievement{\faMusic}{Musique}{Je m'intéresse à de nombreux styles musicaux et je joue du piano.}

  \cvsection{\faTools Compétences}

  \cvtag{Travailleur}
  \cvtag{Autodidacte}
  \cvtag{Persévérant}
  \cvtag{Minutieux}
  \cvtag{Permis de conduire}
  \\

  \divider

  \cvtag{Rust}
  \cvtag{Python}
  \cvtag{C}
  \cvtag{C++}
  \cvtag{Java}
  \cvtag{\LaTeX}
  \cvtag{Linux}
  \cvtag{Systèmes embarqués}

  \cvsection{\faGlobe Langues}

  \textcolor{emphasis}{\textbf{Français}} \hfill \textbf{Langue maternelle}\\

  \divider

  \textcolor{emphasis}{\textbf{Anglais}} \hfill \textbf{Courant (TOEIC 975 - C1)}\\

  \divider

  \textcolor{emphasis}{\textbf{Espagnol}} \hfill \textbf{Basique (A2)}\\

  \divider

  \textcolor{emphasis}{\textbf{Allemand}} \hfill \textbf{Basique (A2)}\\

\end{paracol}

\end{document}
