\documentclass[10pt,a4paper,ragged2e,withhyper]{../AltaCV/altacv}

\geometry{left=1.25cm,right=1.25cm,top=1.5cm,bottom=1.5cm,columnsep=1.2cm}

\usepackage{paracol}
\usepackage{hyperref}

\ifxetex
\setmainfont{Roboto Slab}
\setsansfont{Lato}
\renewcommand{\familydefault}{\sfdefault}
\else
\usepackage[rm]{roboto}
\usepackage[defaultsans]{lato}
\renewcommand{\familydefault}{\sfdefault}
\fi

\definecolor{LightGrey}{HTML}{000000}
\definecolor{Blue}{HTML}{1F7F9F}
\definecolor{Purple}{HTML}{3C3744}
\definecolor{Amber}{HTML}{FFBF00}

\colorlet{name}{Blue}
\colorlet{tagline}{Blue}
\colorlet{heading}{Blue}
\colorlet{headingrule}{Amber}
\colorlet{subheading}{Blue}
\colorlet{accent}{Blue}
\colorlet{emphasis}{Blue}
\colorlet{body}{Purple}

\renewcommand{\namefont}{\Huge\rmfamily\bfseries}
\renewcommand{\personalinfofont}{\footnotesize}
\renewcommand{\cvsectionfont}{\LARGE\rmfamily\bfseries}
\renewcommand{\cvsubsectionfont}{\large\bfseries}

\renewcommand{\cvItemMarker}{{\small\textbullet}}
\renewcommand{\cvRatingMarker}{\faCircle}

\begin{document}
\name{Alix ANNERAUD}
\tagline{Étudiant en informatique de 21 ans à l'INSA de Rouen Normandie.}

\photoL{2cm}{../Images/Alix.png}

\personalinfo{
  \printinfo{\faGlobe}{\underline{\url{https://alix.anneraud.fr/fr}}}
  \email{alix@anneraud.fr}
  \phone{+33 7 81 85 08 69}
  \github{AlixANNERAUD}
  \linkedin{Alix-ANNERAUD}
  \location{Rouen, France}
}

\makecvheader

\columnratio{0.6}

\begin{paracol}{2}
  \cvsection{\faUser À propos}

  \begin{quote}
    ``Actuellement à la recherche d'un \textbf{contrat de professionnalisation} à partir de septembre 2025.
    Je souhaite intégrer une entreprise dynamique pour appliquer mes compétences en ingénierie logiciel et relever des défis techniques concrets.
    ``
  \end{quote}

  \cvsection{\faBriefcase Expériences}

  \cvevent{Stagiaire en ingénierie logicielle}{Amazon Espagne}{Mai -- Août 2025}{Madrid, Espagne}

  \divider

  \cvevent{Stagiaire en recherche}{Université de New York (NYU)}{Juin -- Août 2024}{New York, USA}
  Projet de localisation de sources de gaz (GSL) avec capteurs mobiles et fixes:

  \begin{itemize}
    \item \textbf{Simulations de fluides} et \textbf{algorithmes de localisation}.
    \item \textbf{Traitement du signal} et \textbf{analyse de données} en Python.
    \item Conception et fabrication de capteurs IoT en C++.
  \end{itemize}

  \divider

  \cvevent{Assistant au service des relations entreprises}{INSA de Rouen Normandie}{Septembre 2021 -- Juin 2024}{Rouen, France}

  \textbf{Assistant polyvalent} en gestion de données, relations entreprises et suivi des dossiers étudiants.

  \divider

  \cvevent{Stagiaire sur une ligne de fabrication de rotor et stator}{Renault - Usine Ampère}{Juin 2022 (1 mois)}{Cléon, France}

  \cvsection{\faLightbulb Projets}

  \begin{minipage}[t]{0.49\linewidth}
    \cvevent{\normalsize Xila - \underline{\url{https://xila.dev}}}{}{}{}
    {\small \cvtag{Rust}\cvtag{C} - Système d'exploitation minimaliste et sécurisé pour microcontrôleurs, prenant en charge l’exécution d’applications WASM.}

    \divider

    \cvevent{\normalsize Chef PIC - CHB}{}{}{}
    {\small \cvtag{Vue.js}\cvtag{Django}\cvtag{NLP} - Chef de projet dans une équipe agile pour le développement d'une application web dédiée à la gestion des prescriptions des pharmaciens du Centre Henri Becquerel.}
  \end{minipage}
  \hfill
  \begin{minipage}[t]{0.49\linewidth}
    \cvevent{\normalsize Deez'Nalyzer - \underline{\href{https://github.com/AlixANNERAUD/Deez_Nalyzer}{GitHub}}}{}{}{}
    {\small \cvtag{Vue.js}\cvtag{Django}\cvtag{ML} - Application web pour générer des playlists et classifier des morceaux via l’analyse des MP3 et des habitudes d’écoute avec l’API Deezer.}

    \divider

    \cvevent{\normalsize Cartographie - \underline{\href{https://github.com/AlixANNERAUD/Evidential_occupancy_map}{GitHub}}}{}{}{}
    {\small \cvtag{C++}\cvtag{ROS} - Implémentation d’un algorithme de cartographie évidentielle (EOGM) basé sur la théorie de Dempster-Shafer, optimisé pour le temps réel.}
  \end{minipage}

  \medskip

  \switchcolumn

  \cvsection{\faGraduationCap Formation}

  \cvevent{Diplôme d'ingénieur en informatique}{INSA de Rouen Normandie}{Sept 2021 -- 2026}{Rouen, France}

  \divider

  \cvevent{Baccalauréat scientifique (section européenne)}{Lycée Jean-Paul II}{2021}{Rouen, France}
  Mention très bien.
  
  \divider
  
  \cvevent{Cambridge B1 Preliminary}{}{2018}{}
  
  \cvsection{\faGlobe Langues}

  \textcolor{emphasis}{\textbf{Français}} \hfill \textbf{Langue maternelle}\\

  \divider

  \textcolor{emphasis}{\textbf{Anglais}} \hfill \textbf{Courant (TOEIC 975 - C1)}\\

  \divider

  \textcolor{emphasis}{\textbf{Espagnol}} \hfill \textbf{Basique (A2)}\\

  \divider

  \textcolor{emphasis}{\textbf{Allemand}} \hfill \textbf{Basique (A2)}\\

  \cvsection{\faTools Compétences}

  \cvtag{Travailleur}
  \cvtag{Autodidacte}
  \cvtag{Persévérant}
  \cvtag{Minutieux}
  \cvtag{Permis de conduire}
  \\

  \divider

  \cvtag{Rust}
  \cvtag{C/C++}
  \cvtag{Python}
  \cvtag{\LaTeX}
  \cvtag{Java}
  \cvtag{Django}
  \cvtag{Vue.js}
  \cvtag{Linux}
  \cvtag{Deep learning}
  \cvtag{Traitement d'images}
  \cvtag{Systèmes embarqués}
  \cvtag{Docker}
  \cvtag{CI/CD}


  \cvsection{\faThumbsUp Loisirs}

  \cvachievement{\faMicrochip}{Informatique et électronique}{Passionné par l'informatique et l'électronique depuis l'enfance, je souhaite devenir ingénieur dans ce domaine.}

  \divider

  \cvachievement{\faSubway}{Modélisme}{Construction d'un réseau de trains miniatures avec mon père.}

\end{paracol}

\end{document}
